\nonstopmode{}
\documentclass[a4paper]{book}
\usepackage[times,inconsolata,hyper]{Rd}
\usepackage{makeidx}
\usepackage[utf8,latin1]{inputenc}
% \usepackage{graphicx} % @USE GRAPHICX@
\makeindex{}
\begin{document}
\chapter*{}
\begin{center}
{\textbf{\huge Package `p.exact'}}
\par\bigskip{\large \today}
\end{center}
\begin{description}
\raggedright{}
\item[Type]\AsIs{Package}
\item[Title]\AsIs{Exact P-values for Genome-Wide Association Analyses in Inbred Populations}
\item[Version]\AsIs{0.1-0}
\item[Date]\AsIs{2015-02-04}
\item[Author]\AsIs{Xia Shen}
\item[Maintainer]\AsIs{Xia Shen }\email{xia.shen@ki.se}\AsIs{}
\item[Description]\AsIs{This package implements for calculating exact dataset-specific
p-values for genome-wide association (GWA) analyses in inbred populations.
The module is compatible with existing GenABEL/gwaa.data data formats.}
\item[Depends]\AsIs{R (>= 2.10), GenABEL, svMisc}
\item[License]\AsIs{GPL (>= 2)}
\item[LazyLoad]\AsIs{yes}
\end{description}
\Rdcontents{\R{} topics documented:}
\inputencoding{utf8}
\HeaderA{arab}{Arabidopsis thaliana data set from Atwell et al. 2010 \emph{Nature}}{arab}
\keyword{datasets}{arab}
%
\begin{Description}\relax
The `arab' data contains 107 phenotypes of Arabidopsis and genotypes from a 250K SNP chip as GenABEL gwaa.data format. 
\end{Description}
%
\begin{Format}
 The `Arabidopsis' data set contains a gwaa.data object. See package GenABEL.)
\end{Format}
%
\begin{Source}\relax
Atwell, S., Y. S. Huang, B. J. Vilhjalmsson, G. Willems, M. Horton, et al., 2010. Genome-wide association study of 107 phenotypes in Arabidopsis thaliana inbred lines. \emph{Nature} 465: 627-631.
\end{Source}
%
\begin{References}\relax
Atwell, S., Y. S. Huang, B. J. Vilhjalmsson, G. Willems, M. Horton, et al., 2010. Genome-wide association study of 107 phenotypes in Arabidopsis thaliana inbred lines. \emph{Nature} 465: 627-631.
\end{References}
\inputencoding{utf8}
\HeaderA{p.exact}{Exact p-values}{p.exact}
\aliasA{p.exact-package}{p.exact}{p.exact.Rdash.package}
\keyword{package}{p.exact}
%
\begin{Description}\relax
p.exact: Exact P-values for Genome-Wide Association Analyses in Inbred Populations
\end{Description}
%
\begin{Details}\relax
The p.exact package implements for calculating exact dataset-specific p-values for
genome-wide association (GWA) analyses in inbred populations. The module is
compatible with existing GenABEL/gwaa.data data formats.

For converting data from other formats, see

\code{\LinkA{convert.snp.illumina}{convert.snp.illumina}} (Illumina/Affymetrix-like format).
\code{\LinkA{convert.snp.text}{convert.snp.text}} (conversion from human-readable GenABEL format),
\code{\LinkA{convert.snp.ped}{convert.snp.ped}} (Linkage, Merlin, Mach, and similar files),
\code{\LinkA{convert.snp.mach}{convert.snp.mach}} (Mach-format),
\code{\LinkA{convert.snp.tped}{convert.snp.tped}} (from PLINK TPED format),
\code{\LinkA{convert.snp.affymetrix}{convert.snp.affymetrix}} (BRML-style files).

For converting of GenABEL's data to other formats, see
\code{\LinkA{export.merlin}{export.merlin}} (MERLIN and MACH formats),
\code{\LinkA{export.impute}{export.impute}} (IMPUTE, SNPTEST and CHIAMO formats),
\code{\LinkA{export.plink}{export.plink}} (PLINK format, also exports phenotypic data).

To load the data, see \code{\LinkA{load.gwaa.data}{load.gwaa.data}}.

For data managment and manipulations see
\code{\LinkA{merge.gwaa.data}{merge.gwaa.data}},
\code{\LinkA{merge.snp.data}{merge.snp.data}},
\code{\LinkA{gwaa.data-class}{gwaa.data.Rdash.class}},
\code{\LinkA{snp.data-class}{snp.data.Rdash.class}},
\code{\LinkA{snp.names}{snp.names}},
\code{\LinkA{snp.subset}{snp.subset}}.
\end{Details}
%
\begin{Author}\relax
Xia Shen
\end{Author}
%
\begin{References}\relax
If you use p.exact package in your analysis, please cite the following work:

Xia Shen (2015).
Beyond permutation test: calculating exact dataset-specific p-values for
genome-wide association studies in inbred populations. \emph{Submitted}.
\end{References}
%
\begin{SeeAlso}\relax
\code{GenABEL}
\end{SeeAlso}
\inputencoding{utf8}
\HeaderA{p.exact.binary}{Exact p-values for genome-wide association analysis of case-control data in inbred populations}{p.exact.binary}
\aliasA{p.exact}{p.exact.binary}{p.exact}
\keyword{discovery}{p.exact.binary}
\keyword{exact}{p.exact.binary}
\keyword{false}{p.exact.binary}
\keyword{genome-wide}{p.exact.binary}
\keyword{p-value,}{p.exact.binary}
\keyword{p.exact,}{p.exact.binary}
\keyword{rate}{p.exact.binary}
%
\begin{Description}\relax
The function imports GenABEL (gwaa.data class) data format
and calculates the exact dataset-specific p-values of a case-control
phenotype for each variant or a given odds ratio and allele frequency.
\end{Description}
%
\begin{Usage}
\begin{verbatim}
p.exact.binary(pheno, gwaa.object, or = NULL, or.maf = NULL,
  low.maf = 0.05, high.ld = 0.9, method = "logOR", type = "two-sided",
  con.table = NULL)
\end{verbatim}
\end{Usage}
%
\begin{Arguments}
\begin{ldescription}
\item[\code{pheno}] A string that gives the binary phenotype name in \code{gwaa.object} or
a vector that gives phenotypic values match the order in \code{gwaa.object}.

\item[\code{gwaa.object}] An object of \code{\LinkA{gwaa.data-class}{gwaa.data.Rdash.class}} to be analyzed.

\item[\code{or}] An (optional) vector gives the odds ratios to be tested, and \code{or.maf} must
also be given if not testing for the whole genome.

\item[\code{or.maf}] An (optional) vector gives the minor allele frequencies in accordance with
\code{or}, must be used together with \code{or}. If \code{or.maf} or \code{or}
is missing, exact p-values will be calculated for the whole genome.

\item[\code{low.maf}] A numeric value that gives the cut-off of the lowest minor allele frequency
allowed in the analysis.

\item[\code{high.ld}] A numeric value that gives the cut-off of the highest linkage disequilibrium R-square
allowed in the analysis, i.e. LD pruning.

\item[\code{method}] A string tells the method used, currently only "logOR" is available.

\item[\code{type}] A string tells the statistical test type, can be \code{'one-sided'} or \code{'two-sided'}.

\item[\code{con.table}] Genome-wide contingency tables. If \code{NULL}, to be calculated based on data.
If a string tells the file name, load from the file or to be calculated and saved into the file. If an R
matrix, regard as genome-wide contingency tables (see the saved R object for the format).
\end{ldescription}
\end{Arguments}
%
\begin{Value}
The function returns a data frame of exact p-values (\code{\$p.exact}) and corresponding odds
ratios \code{\$or} and MAFs \code{\$MAF}.
\end{Value}
%
\begin{Note}\relax
None.
\end{Note}
%
\begin{Author}\relax
Xia Shen
\end{Author}
%
\begin{References}\relax
Xia Shen (2015). Beyond permutation test: calculating exact dataset-specific p-values
for genome-wide association studies in inbred populations. \emph{Submitted}.
\end{References}
%
\begin{SeeAlso}\relax
\code{\LinkA{ccfast}{ccfast}}, \code{\LinkA{glm}{glm}}
\end{SeeAlso}
%
\begin{Examples}
\begin{ExampleCode}
## Not run: 
## loading example gwaa.data of data from Atwell et al. (2010) Nature
data(arab)

## running a regular GWA analysis for AvrRPM1
cc1 <- ccfast('X33_.i.avrRpm1..i.', data = arab)

## check the top finding using the exact p-value
top <- which.min(cc1[,'P1df'])
or <- cc1[top,'effB']
f <- summary(arab[,top])$Q.2
maf <- min(f, 1 - f)
exact <- p.exact.binary(pheno = 'X33_.i.avrRpm1..i.', gwaa.object = arab, or = or, or.maf = maf, con.table = 'tab.AvrRPM1.RData')

## End(Not run)
\end{ExampleCode}
\end{Examples}
\inputencoding{utf8}
\HeaderA{p.exact.gaussian}{Exact p-values for genome-wide association analysis in inbred populations}{p.exact.gaussian}
\aliasA{p.exact}{p.exact.gaussian}{p.exact}
\keyword{discovery}{p.exact.gaussian}
\keyword{exact}{p.exact.gaussian}
\keyword{false}{p.exact.gaussian}
\keyword{genome-wide}{p.exact.gaussian}
\keyword{p-value,}{p.exact.gaussian}
\keyword{p.exact,}{p.exact.gaussian}
\keyword{rate}{p.exact.gaussian}
%
\begin{Description}\relax
The function imports GenABEL (gwaa.data class) data format
and calculates the exact dataset-specific p-values for each variant
or a given effect size and allele frequency.
\end{Description}
%
\begin{Usage}
\begin{verbatim}
p.exact.gaussian(gwaa.object = NULL, n = NULL, maf = NULL, beta,
  beta.maf = NULL, low.maf = 0.05, type = "two-sided")
\end{verbatim}
\end{Usage}
%
\begin{Arguments}
\begin{ldescription}
\item[\code{gwaa.object}] An (optional) object of \code{\LinkA{gwaa.data-class}{gwaa.data.Rdash.class}}.

\item[\code{n}] An (optional) integer gives the sample size, only used when \code{gwaa.object = NULL}.

\item[\code{maf}] An (optional) vector gives minor allele frequencies across the genome,
only used when \code{gwaa.object = NULL}.

\item[\code{beta}] An vector gives the effect sizes to be tested, and \code{beta.maf} must
also be given if not testing for the whole genome.

\item[\code{beta.maf}] An (optional) vector gives the minor allele frequencies in accordance with
\code{beta}, must be used together with \code{beta}. If \code{beta.maf}
is missing, exact p-values will be calculated for the whole genome.

\item[\code{low.maf}] A numeric value that givens the cut-off of the lowest minor allele frequency
allowed in the analysis.

\item[\code{type}] A string tells the statistical test type, can be \code{'one-sided'} or \code{'two-sided'}.
\end{ldescription}
\end{Arguments}
%
\begin{Value}
The function returns a data frame of exact p-values (\code{\$p.exact}) and corresponding effect
sizes \code{\$beta} and MAFs \code{\$MAF}.
\end{Value}
%
\begin{Note}\relax
None.
\end{Note}
%
\begin{Author}\relax
Xia Shen
\end{Author}
%
\begin{References}\relax
Xia Shen (2015). Beyond permutation test: calculating exact dataset-specific p-values
for genome-wide association studies in inbred populations. \emph{Submitted}.
\end{References}
%
\begin{SeeAlso}\relax
\code{\LinkA{qtscore}{qtscore}}, \code{\LinkA{t.test}{t.test}}
\end{SeeAlso}
%
\begin{Examples}
\begin{ExampleCode}
## Not run: 
## loading example gwaa.data in GenABEL
data(ge03d2ex.clean)

## running a regular GWA analysis for height
qt1 <- qtscore(height, data = ge03d2ex.clean)

## running the multivariate GWAS again
exact <- p.exact.gaussian(ge03d2ex.clean, beta = qt1[,'effB'])

## End(Not run)
\end{ExampleCode}
\end{Examples}
\printindex{}
\end{document}
